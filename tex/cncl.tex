\section{Conclusions}
\label{sec:cncl}

In this study, we addressed the issue of memory accesses collection for
multithreaded applications. This is a key challenge in high performance
computing as memory is often a performance
bottleneck. Memory traces can be used at runtime to improve data locality or
offline by developers to understand and improve their applications memory
behavior and therefore their performances.

To address this challenge, we have proposed \Moca an efficient memory trace
collection system. While other existing tools rely on \emph{incomplete} sampling to
provide such traces at an acceptable cost, \Moca provides a \emph{complete}
trace, that contains all the accessed areas, at the granularity of the page.
Moreover, \Moca traces not only
contain all pages that are accessed during the execution, but also, 
for each access, they contain temporal, spacial and sharing
information: which thread, accessed what addresses on which CPU and when.
While \Moca works at the page granularity, it stores the exact
address of each intercepted accesses. Therefore, it also provide an
\emph{incomplete} trace at the granularity of the byte, similar to
traces collected by instructions sampling.

When evaluated along with a comparable tool based on binary instrumentation,
\Moca demonstrates a significantly lower overhead despite the fact that it collects
much more detailed traces.

Future work will focus on the visualization and exploitation of these memory traces
which is another challenge mainly due to the volume of the collected data.
