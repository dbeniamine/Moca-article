\section{Conclusions}
\label{sec:cncl}

In this study, we addressed the issue memory accesses collection. This is a
very important challenge in high performance architecture as memory is often a
bottleneck. Memory traces can be used at runtime to improve data locality or
offline by developers to understand and improve their applications memory
behavior and therefore their performances.

To address this challenge, we  presented \Moca an efficient memory trace
collection system, while other existing tools relies on \emph{incomplete} sampling to
provide such traces at an acceptable cost, \Moca provides a \emph{complete}
trace at the granularity of the page. By this we mean that \Moca traces
contains every pages that are accessed during the execution at least once,
and, for each accesses, it contains temporal, spacial and sharing
informations (which thread, accessed what addresses on which CPU and when).
Furthermore while \Moca works at the page granularity, it store the exact
address of each intercepted accesses, therefore it provide an
\emph{incomplete} trace at the granularity of the byte.

While \Moca is highly tunable, we evaluated with its default parameters by
comparing it to existing tools in termes of performances and trace details.
Our evaluations shows that with its default settings, \Moca is faster in most
cases and at worst as slow as existing tools to collect more detailed memory
traces.

Future work will focus on the visualization and exploitation of memory traces
which is a challenge as those traces are rich in terms of data.
