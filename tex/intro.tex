\section{Introduction}
\label{sec:intro}

In \emph{High Performance Computing} memory is often a bottleneck.
While it is trivial to access it, doing it efficiently is extremely
complex, indeed the user needs to take into account the cache hierarchy and
their mechanisms such as the pre-fetcher. It gets even more complex with
\emph{Not Uniform Memory Access} machines where the location of data in
physical memory impacts the performances~\cite{Drepper07What}.

Several tools such as NUMA Balancing~\cite{Corbet2012} were designed to improve the
memory usage on NUMA machines. These tools are able to move data on the
physical memory to keep it as close as possible to the thread using them
reducing the number of remote accesses. These tools provides great results
still they cannot fix every memory related issues. For instance if every
threads of an application accesses to the same page, every page mapping will
results on remote accesses from every NUMA node but one. The only way to fix
this kind of issues is to rewrite the application's code considering the
memory pattern. Thus tools helping the user to understand the memory pattern
of its application can help improving performances.

Numerous tools such as Vtune~\cite{Reinders05VTune} or
HPCToolkit~\cite{Adhianto10HPCTOOLKIT} where developed to help the programmer
understanding the performances issues of its applications. However these tools
focuses on the CPU and can only provides a partial view of events happening on
the memory. Thus they are not able to help solving memory access pattern
related issues.

To solve this kind of issue, the user needs a \emph{complete} trace of memory
accesses with a fine \emph{granularity}. We consider that a trace is
\emph{complete} at a certain granularity if and only if it provides a superset
of accesses at this granularity with for each access temporal, spacial and
thread related information.

Memory analysis tools are currently not able to provide such traces, they
always give up some information by doing some \emph{incomplete}
sampling~\cite{Liu14Tool,Lachaize12MemProf}, increasing granularity and/or
removing temporal information~\cite{Beniamine15TABARNACRR}.

Generating such traces is a challenge, indeed there is not hardware mechanism
comparable to CPU performance counters to trace memory access, and the amount
of data to collect is considerable. Instrumentation based methods are too slow
to provide a complete trace, and efficient hardware sampling mechanisms cannot
provide such trace by conception.

I this study, we present \emph{Memory Organisation, Cartography and Analysis}
an efficient memory trace collection system based on page fault interception.
This mechanism is able to provide a \emph{complete} sampled memory trace at
the granularity of the page with information on accesses at the granularity of
the address.

% \begin{itemize}
%     \item HPC => importance of memory
%         \begin{itemize}
%             \item NUMA
%             \item Caches
%         \end{itemize}
%     \item Runtimes aren't the only solution
%         \begin{itemize}
%             \item Overhead
%             \item Only provide ``good'' page mapping
%             \item Sometimes provides worts results
%             \item Can't fix bad patter
%         \end{itemize}
%     \item  Need of memory profiling
%         \begin{itemize}
%             \item Granularity: granularity of the stored information
%             \item Complete trace: trace that cannot miss a part of the memory
%                 (can miss events)
%         \end{itemize}
%     \item Existing tools:
%         \begin{itemize}
%             \item Most are CPU oriented
%             \item Few Memory oriented
%                 \begin{itemize}
%                     \item Show very precise informations (number of remote
%                         access)
%                     \item Often relies on info from CPU
%                     \item No global view of the memory
%                     \item No temporal informations
%                     \item Either high granularity or not complete
%                 \end{itemize}
%         \end{itemize}
%     \item Getting a global view of the memory access over time is hard:
%         \begin{itemize}
%             \item How to collect efficiently
%             \item Lot of data to store / keep on memory
%             \item Times means synchronisations
%         \end{itemize}
%     \item Conclusion
%         \begin{itemize}
%             \item Efficient collection
%             \item Complete trace at the page granularity
%             \item Sampled information at the address granularity
%         \end{itemize}
% \end{itemize}
