\section{Introduction}
\label{sec:intro}

In \emph{High Performance Computing} the memory subsystem is often a performance bottleneck.
Using efficiently this memory subsystem can be extremely
complex, indeed the developer has to take into account both the cache hierarchy and
hardware mechanisms such as the memory prefetcher. It gets even more complex with
\emph{Non Uniform Memory Access} machines in which data location in
physical memory impacts the accesses latency and contention issues arise~\cite{Drepper07What}.

Most efforts about memory performance optimizations at software level consists
on tools such as NUMA Balancing~\cite{Corbet2012}.
These tools automatically moves pages of data in the physical memory, trying
to keep them as close as possible from the threads that use them. Such
optimizations can improve significantly the memory performances as it reduces
the number of remote accesses. Still this cannot address all the memory related issues. For instance, if all the
threads of an application access to the same memory page, whatever the page mapping is,
this will result in remote accesses from all NUMA nodes but one. The only way to fix
this kind of issues is to rewrite a part of the application code in order to take into account these
unbalanced memory accesses. This is why tools that can help the developer to understand the memory accesses patterns
of its application are of utter importance for performances optimizations.

Analysis tools such as Vtune~\cite{Reinders05VTune} and
HPCToolkit~\cite{Adhianto10HPCTOOLKIT} have been developed to help the programmer
understand and correct performance issues in a multithreaded application. However, these tools
focus on the CPU and can only provides a indirect and incomplete view of events related to the memory: the location of accesses,
their time or both are usually lost in the process. Thus, they are not able to provide the developer with
a clear view of memory accesses patterns occurring during the execution. In such situation, fixing
memory related issues is a matter of trial and error and the eventual code is often suboptimal.

Indeed, to solve memory related issues, the developer needs a detailed trace of memory
accesses at a sufficiently fine \emph{granularity}.
An ideal tool should provide enough data to recreate a cartography view
of the memory accesses over the time.
\DB{Add sketch of visu ?}
Moreover, to ensure that a lack of precision
does not compromise the analysis, such a trace should be \emph{complete}. We say that a trace is
\emph{complete} at a certain granularity if and only if the events it contains along with their granularity
form a superset of the actual accesses. Furthermore, memory events in a \emph{complete} trace should
include information about time, space (at which address the event occurs),
location (on which CPU it occurs) and nature of access (is this
a read, a write, by which thread).
At the time of this writing, existing memory analysis tools are not able to provide such traces.
Either they rely on \emph{incomplete} sampling~\cite{Liu14Tool,Lachaize12MemProf}
mechanisms and thus only provide a very small subset of the trace, or they
ignore temporal information to reduce their
overhead~\cite{Beniamine15TABARNACRR}.

Generating such traces is a challenge: there is no hardware mechanism
comparable to CPU performance counters to collect a detailed trace of memory accesses, and the volume
of data to collect is huge. Instrumentation based methods are too slow
to provide a \emph{complete} trace, and efficient hardware sampling mechanisms are not designed
to provide all the information that constitute a \emph{complete} trace.

In this study, we present \emph{Memory Organisation Cartography and Analysis}
\footnote{\Moca is distributed under GPL licence:\\
    \url{https://github.com/dbeniamine/MOCA}}
an efficient memory trace collection system based on page fault interception
and false page fault generation.
This mechanism is able to provide a \emph{complete} sampled memory trace at
the spatial granularity of the page along with a parametrized temporal granularity.
It also collects detailed information about sampled accesses, including their address and
precise timestamp. Finally \Moca is able to retrieve data structures
information (address, size and name) using a binary instrumentation.

The remaining of this paper is organised as follow: section~\ref{sec:related}
discuss related works, section~\ref{sec:design} present \Moca design, then we
evaluate \Moca by comparing it to existing tools on section~\ref{sec:expe}.
Finally we present our conclusions and some possible future work on
section~\ref{sec:cncl}.
