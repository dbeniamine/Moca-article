\section{Introduction}
\label{sec:intro}

In \emph{High Performance Computing} the memory subsystem is often a performance bottleneck.
Using efficiently this memory subsystem can be extremely
complex, indeed the developer has to take into account both the cache hierarchy and
hardware mechanisms such as the memory prefetcher. It gets even more complex with
\emph{Non Uniform Memory Access} machines in which data location in
physical memory impacts the accesses latency and contention issues arise~\cite{Drepper07What}.

Most efforts about memory performance optimizations at software level consists
in tools such as NUMA Balancing~\cite{Corbet2012}.
These tools automatically moves pages in the physical memory, trying
to keep them as close as possible from the threads that use them. Such
optimizations can improve significantly the memory performances as they reduce
the number of remote accesses. Still they cannot fix improper algorithms and data structures that do not make an efficient use of memory resources.
For instance, if all the
threads of an application access to the same memory page, whatever the page mapping is,
this will result in remote accesses from all NUMA nodes but one. The only way to fix
this kind of issues is to rewrite a part of the application code in order to take into account these
unbalanced memory accesses. This is why tools that can help the developer understand the memory accesses patterns
performed by his application are of utter importance for performance optimizations.

Analysis tools such as Vtune~\cite{Reinders05VTune} and
HPCToolkit~\cite{Adhianto10HPCTOOLKIT} have been developed to help the programmer
understand and correct performance issues in a multithreaded application. These tools
mostly rely on hardware performance counters of the CPU and, as such, can only trace an indirect and incomplete view of events related to the memory:
the location of accesses,
their time or both are usually lost in the process. Thus, they are not able to provide the developer with
a clear view of memory accesses patterns occurring during the execution. In such situation, fixing
memory related issues is a matter of trial and error and the eventual code is often suboptimal.

To help the developer solve memory related issues, an ideal tool should provide enough data to build a map of the memory accesses locations over the time.
Therefore, events in a memory trace should include information about time, space (at which address the event occurs),
location (on which CPU it occurs) and nature of access (is this
a read, a write, by which thread), we name such trace a \emph{detailed} trace.
The trace has to be \emph{precise} enough, it should include a sufficiently large number of events in order to enable a sound analysis.
Moreover, to ensure that a lack of precision
does not compromise the analysis, such a trace should be \emph{complete}. We say that a trace is
\emph{complete} at a given granularity if and only if the events it contains
form a superset of the actual accesses.
For instance, \Moca, the tool we present in this article, collects \emph{complete} traces at the page granularity.
At the time of this writing, existing memory analysis tools are not able to provide \emph{detailed} traces that are \emph{complete} and sufficiently \emph{precise}.
Some of them rely on instructions sampling~\cite{Liu14Tool,Lachaize12MemProf}
mechanisms, which produce \emph{incomplete} \emph{detailed} traces at a precision limited by hardware capabilities. Other ones ignore temporal information to reduce their
overhead and collect non \emph{detailed} traces~\cite{Beniamine15TABARNACRR}.

Indeed, generating such traces is a challenge: there is no hardware mechanism
comparable to CPU performance counters to collect a \emph{detailed} trace of memory accesses, and the volume
of data to collect is huge. Methods based on binary instrumentation are too slow
to provide a \emph{complete} \emph{detailed} trace, and efficient hardware sampling mechanisms are not designed
to collect all the events required to constitute a \emph{complete} trace and do not collect at an arbitrary \emph{precision}.

In this study, we present \emph{Memory Organisation Cartography and Analysis}%
\footnote{\Moca is distributed under GPL licence:
    \href{https://github.com/dbeniamine/MOCA}{github.com/dbeniamine/MOCA}},
an efficient memory trace collection system based on two existing techniques:
page faults interception and false page faults injection.
This tool is able to collect a \emph{complete} \emph{detailed} memory trace at
the spatial granularity of the page and at a parametrized temporal granularity.
It collects \emph{detailed} information about a particular sample of all the accesses, including the addresses and
timestamp of these accesses. Furthermore, \Moca is able to retrieve associated data structures
information (address, size and name) using the data contained in the application binary.

The rest of this paper is organised as follows: in section~\ref{sec:related}
we discuss about related works, in section~\ref{sec:design} we present the \Moca design, then,
we evaluate \Moca by comparing it to existing tools in section~\ref{sec:expe}.
Finally, we present our conclusions and some possible future work in
section~\ref{sec:cncl}.
