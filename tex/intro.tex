\section{Introduction}
\label{sec:intro}


In \emph{High Performance Computing} memory is often a bottleneck while it is
trivial to access it, it is more then complex to do it efficiently. Several
automated tools were designed to improve the memory usage on \emph{Not Uniform
Memory Access} machines. These tools are able to move data on the memory to
keep it as close as possible to the thread using them. Still they cannot
change the memory access pattern. Thus the developer has to consider this
pattern to produce efficient code.

Numerous tools are able to generate a trace of events happening to the
CPU during an execution and display them in a comprehensible way. Only a few
tools address this problem from the memory point of view, and most of the time
they focus on a small subset of event such as NUMA remote accesses.

Providing a trace of the memory accesses of an application over time is
fundamental to help the developer apprehending it's application memory pattern
and improve it.

%Generating such trace is challenging \ldots
\DB{Continue \ldots}


\begin{itemize}
    \item HPC => importance of memory
        \begin{itemize}
            \item NUMA
            \item Caches
        \end{itemize}
    \item Runtimes aren't the only solution
        \begin{itemize}
            \item Overhead
            \item Only provide ``good'' page mapping
            \item Sometimes provides worts results
            \item Can't fix bad patter
        \end{itemize}
    \item  Need of memory profiling
    \item Existing tools:
        \begin{itemize}
            \item Most are CPU oriented
            \item Few Memory oriented
                \begin{itemize}
                    \item Show very precise informations (number of remote
                        access)
                    \item Often relies on info from CPU
                    \item No global view of the memory
                    \item No temporal informations
                \end{itemize}
        \end{itemize}
    \item Getting a global view of the memory access over time is hard:
        \begin{itemize}
            \item How to collect efficiently
            \item Lot of data to store / keep on memory
            \item Times means synchronisations
        \end{itemize}
\end{itemize}
