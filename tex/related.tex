%!TEX encoding=UTF-8 Unicode

\section{Related Works}
\label{sec:related}

Several generic tools have been designed to analyze and improve parallel
applications performances, such as Intel's VTune~\cite{Reinders05VTune},
Performance Counter Monitor~(PCM)~\cite{Intel2012b}, the
HPCToolkit~\cite{Adhianto10HPCTOOLKIT}, and AMD's
CodeAnalyst~\cite{Drongowski2008}. All of these tools use performance
counters and execution traces to show when and where CPUs are idle, and, thus,
highlight potential places for improvements.
But they mostly focus on CPU related information and provide only
indirect data about memory performances, ones that can be computed from the performance counters values.

As performances counters are architecture dependent and are not always easily
understandable, higher level libraries such as \emph{PAPI}~\cite{Weaver13PAPI} and
\emph{Likwid}~\cite{Treibig10LIKWID} have been developed to ease their analysis. These
libraries are able to derive more abstract and understandable metrics from raw hardware counters.
For instance Likwid provides several groups of metrics related to memory
providing the user with computed bandwidth in each cache level and between each CPU core
and the main memory.
Overall, a lot of studies provide a memory analysis solely based
on information collected through these hardware performance counters~\cite{Majo13(Mis)understanding,
Jiang14Understanding,Bosch00Rivet,Weyers14Visualization,Tao01Visualizing,DeRose01Hardware}.
Despite providing a good global summary about memory performance figures, these counters only provide a partial view of the execution. They
account for memory related events from the point of view of one processor, but
do not give a precise insight about the place and the cause of these events.

Another approach used by several
tools~\cite{Lachaize12MemProf,McCurdy10Memphis,Liu14Tool,Gimenez14Dissecting}
consists in using instructions sampling mechanisms such as AMD's Instruction Based Sampling
(IBS)~\cite{Drongowski07Instructionbased} or Intel Precise Event Based
Sampling (PEBS)~\cite{Levinthal2009} to trace the application execution. These methods
provide \emph{incomplete} sampling: some parts of the memory can be accessed without
being noticed by the tool if none of the associated instructions are part of the sampled
instructions.
Thus, it is possible that they ignore some parts of the memory less frequently accessed but in which
optimization could take place.
Despite their low frequency, application sensitive to spurious performance degradation, such as interactive applications, could be hindered by these unnoticed
accesses.

To make things practical, these sampling mechanisms monitor what they name an events set given by an instruction type along with some predicates.
They can monitor several events sets at the same time but the number of monitored sets is limited by the hardware capabilities (number of available
registers). Unfortunately, the number of existing events sets that relate to the memory hierarchy is large, because of its complexity.
This makes difficult the task of tracing all the relevant memory accesses with just a single analysis.
One way to lessen the impact of this limitation is to run several times the
instrumentation and use advanced methods such as
folding~\cite{Servat15Towards} to generate a more accurate summary trace.
Nevertheless, this makes the instrumentation cost grow accordingly.
Moreover, writing (and sometimes) using tools that relies on hardware mechanisms
requires a deep knowledge of the processor. As processors evolve,
such tools are hard to maintain and can quickly become outdated.
We regard all these limitations as too constraining for a general purpose
memory analysis tool.

Some other studies make use of hardware modification, either actual or
simulated~\cite{Bao08HMTT,Martonosi92MemSpy}.  Although they are eventually able to collect
more precise traces efficiently, these techniques are limited to hardware
developers. Indeed, to use these hardware extensions one has either to obtain (or build) a
prototype or to use a suitable simulator. Such configuration is not realistic
for general purpose memory analysis.

Binary instrumentation can provide accurate information about memory accesses.
This method is portable and  more precise than the aforementioned ones,
but it comes at the cost of performances. 
% Référence supprimée: trop expeditive, SIGMA mélange une vrai instrumentation
% avec des simulation a partir de la trace pour extrapoler le comportement sur
% d'autres architectures, c'est un peu trop loin je pense
% Fast
% instrumentations usually either rely on a simulator~\cite{DeRose02SIGMA} which is not as realistic as an actual execution
% \GH{simulation ou emulation ? Pas réaliste ? il faut un peu argumenter..}
Tools that rely on binary instrumentation either collect data at a coarser
granularity and give up temporal data~\cite{Beniamine15TABARNAC} or merge
accesses into higher level representation and do online compression of the
trace~\cite{Budanur11Memory} to limit their overhead.


Page faults interception can provide useful online information about memory usage,
such a mechanism has been used in several existing works : in parallel
garbage collectors~\cite{Boehm91Mostly}, in memory
checkpointing~\cite{Heo05Spaceefficient} or in the domain of virtualization to
provide the hypervisor with information about the memory usage of the guest
OS~\cite{Jones06Geiger}. Nevertheless, page faults only occur when caused by predetermined
events in the system. Thus, just intercepting existing page faults only provide a broad view of the
memory behavior. To reach a deeper understanding of this memory behavior,
it is also possible to fake invalid pages at regular intervals in order to generate false faults~\cite{Bae12Dynamic,Diener13CommunicationBased}.
These false page faults are just triggered during regular memory accesses that would not have 
caused a page fault if the page were not faked as invalid. The advantage is that they create additional
events for the monitoring tool to collect, thus more precision, but the set of faked invalid pages has to be known and maintained by the monitoring tool.

As a final note, most tools close to our proposal do not use false page faults injection and only need to store the location of memory pages and the threads that access them.
As a consequence, they require a relatively small data structure in memory for their own usage.
In this study we present \Moca, a new \emph{complete} memory trace collection system, based on page
fault interception and false page faults injection, able to capture precisely the temporal evolution of multithreaded applications memory accesses.
To reach a satisfying precision, our tool has to maintain in memory both the trace data and
the set of faked invalid pages. Overall, storing and exploiting efficiently these data within the kernel space and outputting it in real time to the user space
is a challenge and is the main contribution of our work.
